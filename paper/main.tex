% !TEX program = xelatex
\documentclass[twocolumn,superscriptaddress,english,showpacs,longbibliography]{revtex4-2}
\usepackage[colorlinks=true,urlcolor=blue,citecolor=blue,linkcolor=blue]{hyperref} 

\usepackage{svg}
\usepackage{amsmath}
\usepackage{graphicx}% Include figure files
\usepackage{textcomp}
\usepackage{bm}% bold math
\usepackage{color}
\usepackage{amssymb}
\usepackage{amsthm}
\usepackage{graphicx}
\usepackage{color}
\usepackage{mathrsfs}
\usepackage{float}
\usepackage{indentfirst}
\usepackage{txfonts}
\usepackage{algorithm}  
\usepackage{algpseudocode}  
\usepackage{balance}
\usepackage{flushend}
\usepackage{cleveref}
\usepackage{tikz}
\renewcommand{\algorithmicrequire}{\textbf{Input:}}  % Use Input in the format of Algorithm  
\renewcommand{\algorithmicensure}{\textbf{Output:}} % Use Output in the format of Algorithm  
\newtheorem{hyp}{Hypothesis}

\newcommand{\jgl}[1]{[{\color{blue}{JGL: #1}}]}
\newcommand{\ym}[1]{[{\color{red}{YM: #1}}]}
\newcommand{\s}{\mathbf {s}}
\newcommand{\net}[1]{{\textsc{#1}}}
\newcommand{\argmin}{\mathop{\mathrm{argmin}}\limits}
\newcommand{\Eq}[1]{Eq.~(\ref{#1})}
\newcommand{\Fig}[1]{Fig.~\ref{#1}}

\begin{document}

\title{Surface programmable materials}

\author{Authors}
\email{xxx@xxx.com}
\affiliation{
    XXX
}

\begin{abstract}
\end{abstract}

\maketitle

\section{Introduction}

\textbf{Definition: (Surface programmable material)}: a lattice (with
translational invariance) model that can be programmed on its surfaces
to perform universal computation.

\jgl{Please cover the following topics: \begin{enumerate}
    \item Computational models driven by the thermodynamics (cooling, or lowering the free energy), i.e. Brownian machine, such as DNA and Ising machine.
    \item Potential advantages: energy saving, parallelism, no size limits (robust to quantum effects). Disadvantages: slow, error-prone.
    \item Two approaches to lower the free energy: entropy driven, or energy based modeling. Embedding a computational model to energy models, the non-deterministic version: Circuit SAT, QUBO, Spin-Glass and Ising machine.
    \item Crutial issue of energy based model, hard to thermalize, hard to prepare.
    \item Crystal based computing: Cellular automata as a computational model.
    \item The method to thermalize, inspired from crystalization and heat zone.
\end{enumerate}
}

Improving the energy efficiency of computation is a major challenge in the development of future computing systems.
The thermodynamic computer (TC)~\cite{conte2019thermodynamic} is a new type of computer that uses thermodynamics to perform computation.
Instead of performing the computation deterministically, TCs are stochastic and are driven by thermodynamics.
Physical implementations of TCs include DNA machines~\cite{tanaka2005design}, memristors~\cite{yang2013memristive, wang2017memristors, kumar2017chaotic, wang2018fully}, and Ising machines~\cite{gu2012encoding, lucas2014ising, mohseni2022ising, fabre2014optical}.

The computational process of TCs are driven by the lowering of free energy, which is a fundamental law of thermodynamics.
A physical system has the tendency to evolve toward the state with the lowest free energy.
By manipulating the minimum free energy state of the system, the computation can be performed.
The change of free energy minimum state can be achieved by changing the distribution of the particles in the system, such as in the case of DNA machines~\cite{Feynman2018} or lowering the temperature of the system, such as in the case of Ising machines~\cite{boixo2013experimental, boixo2014evidence, Pichler2018, Nguyen2023}.
The former is known as the Brownian machine, which evolves the system toward the direction of the entropy increase, while the latter are energy based, which evolves the system toward the direction of the energy decrease.
\jgl{What category are the memristors?}

Energy based models are probability one of the most powerful models for computation.
The Ising model is not only capable of universal computation~\cite{gu2012encoding}, and it can be used to formulate many NP problems~\cite{lucas2014ising,mohseni2022ising}.
However, they are notoriously hard to thermalize, even with the help of quantum entanglement~\cite{boixo2013experimental, boixo2014evidence}.

Intrigued by the question that how to lower the energy of an Ising machine, we propose a new type of TC, which is based on a translational invariant crystalline structure.
We first show that the crystalline structure can be programmed on its surfaces to perform universal computation.
Then we show that if we thermalize the crystal from the ``deterministic direction'', the crystal can thermalize in time almost linear to the size of the crystal, i.e. efficient computation can be performed.

% The way to lower the free energy is inspired by the process of crystalization.
% The material is orderly grown by slowly moving the temperature gradient in-plane~\cite{Zhang2014}.
% Growing a perfect crystal itself is a computation process of copying the information from one surface to another.
%In the DNA based TC, the DNA enzyme is driven by thermodynamics, and the computation is performed by the enzyme as it moves along the DNA strand. It is a Brownian type computational model~\cite{Feynman2018} that driven by the change of entropy.

% TCs are distinguished by their ability to employ the underlying physics of the computing substrate to accomplish a task.
% For example, TCs may employ naturally occurring device-level fluctuations to explore a state space and to stabilize on low-energy representations, which may then be employed in an engineered task.
% TCs are directly connected to real-world potentials, which drive the evolution of their internal organization.
% A user might program constraints that describe an optimization objective over the external potentials and that capture the TC’s natural tendency to maximize entropy production in the environment while minimizing entropy production internally.
% TCs have the potential to be more energy efficient than conventional computers, and they may be able to solve problems that are difficult for conventional computers to solve.
% TCs are still in the early stages of development, but they have the potential to revolutionize computing in the future.

% A thermodynamic computer is a complex system, driven by thermodynamics when exposed to external potentials. Like quantum computers, TCs are distinguished by their ability
% to employ the underlying physics of computing substrate to
% accomplish a task. For example, TCs may employ naturally
% occurring device-level fluctuations to explore a state space
% and to stabilize on low-energy representations, which may
% then be employed in an engineered task~\cite{conte2019thermodynamic}.

% Comparing with all conventional computers fully controlled by humans, the TC, on the other hand, are directly connected to real-world potentials, which drive the
% evolution of its internal organization. As an example of using a TC, a user might program
% constraints that describe an optimization objective
% over the external potentials and that capture the TC’s
% natural tendency to maximize entropy production in the
% environment while minimizing entropy production internally~\cite{conte2019thermodynamic}.

\ym{Need to add a brief summary about the state-of-the-art TCs. Following is some basic summaries of references}

%Quantum annealing outperform classical annealing~\cite{boixo2013experimental, boixo2014evidence}.
%Firstly introduced Rydberg atoms to solve MIS problem with quantum property~\cite{Pichler2018}. Utilized quantum annealing on Rydberg arrays to efficiently solve the combinatorial optimization problem~\cite{Nguyen2023}

%Provided basic knowledge about computational model like DNA Enzyme driven by thermodynamics, raised concepts of reversible computing and gradient power~\cite{Feynman2018}.

The material was orderly grown by slowly moving the temperature gradient in-plane~\cite{Zhang2014}. Tested thermal gradient force for grain boundary migration~\cite{Bai2015}.

Discussed the architecture of future thermodynamic computer and brought up an idea about the understanding of using TC to solve problems~\cite{hylton2021vision}.


%Proved Ising model is capable of universal computation~\cite{gu2012encoding}. Provided Ising formulations of many NP problems~\cite{lucas2014ising}.
%Reviewed various type of Ising machines as hardware analog solvers of combinatorial optimization problems~\cite{mohseni2022ising}.
%A special optical form Ising machine~\cite{fabre2014optical}.
%Other type of TCs like memristors~\cite{yang2013memristive, wang2017memristors, kumar2017chaotic,wang2018fully}, DNA machine~\cite{tanaka2005design}. 




\section{Overview of the computation model}\label{Overview-of-main-results}
In this article, We develop a new type of TC shown in [Fig.\ref{Surface-programmable-thermodynamic computer}(a)], which features a crystalline structure based on an universal elementary cellular automaton, makes it easier to be constructed in real life. 

\tikzset{every picture/.style={line width=0.75pt}} %set default line width to 0.75pt        
\tikzset {_ue4dovf1e/.code = {\pgfsetadditionalshadetransform{ \pgftransformshift{\pgfpoint{0 bp } { -37.5 bp }  }  \pgftransformrotate{0 }  \pgftransformscale{2 }  }}}
\pgfdeclarehorizontalshading{_qxg3a7xt1}{150bp}{rgb(0bp)=(1,0.33,0);
rgb(37.5bp)=(1,0.33,0);
rgb(49.856583731515066bp)=(1,0.19,0);
rgb(62.5bp)=(1,0.33,0);
rgb(100bp)=(1,0.33,0)}
\tikzset{_licxtq9bp/.code = {\pgfsetadditionalshadetransform{\pgftransformshift{\pgfpoint{0 bp } { -37.5 bp }  }  \pgftransformrotate{0 }  \pgftransformscale{2 } }}}
\pgfdeclarehorizontalshading{_h9amsdqzx} {150bp} {color(0bp)=(transparent!80);
color(37.5bp)=(transparent!80);
color(49.856583731515066bp)=(transparent!30);
color(62.5bp)=(transparent!80);
color(100bp)=(transparent!80) } 
\pgfdeclarefading{_htcz8osfa}{\tikz \fill[shading=_h9amsdqzx,_licxtq9bp] (0,0) rectangle (50bp,50bp); } 
\tikzset{every picture/.style={line width=0.75pt}} %set default line width to 0.75pt        

\begin{figure}[h]
    \centering


    
    \begin{tikzpicture}[x=0.75pt,y=0.75pt,yscale=-1,xscale=1, >=stealth]
    %uncomment if require: \path (0,408); %set diagram left start at 0, and has height of 408
    
    %Shape: Axis 2D [id:dp47589449610201395] 
    % \draw  (252,306) -- (464.6,306)(252,45.4) -- (252,306) -- cycle (457.6,301) -- (464.6,306) -- (457.6,311) (247,52.4) -- (252,45.4) -- (257,52.4)  ;
    \draw[->][line width=1.5pt] (257,315) -- (464.6,315) node[anchor=north west] {};
    % 绘制 Y 轴
    \draw[->][line width=1.5pt] (257,315) -- (257,45.4) node[anchor=south east] {};
    % \node at (358.3, 306) [anchor=north] {$Time$};
    \draw (358.3,320) node [anchor=north][inner sep=0.75pt]   [align=left] {${Time}$};
    % \node at (252, 175.7) [anchor=east] {$Space$};
    \draw (215,175.7) node [anchor=west][inner sep=0.75pt]   [align=left] {${Space}$};
    %Rounded Rect [id:dp5816101232993451] 
    \path  [shading=_qxg3a7xt1,_ue4dovf1e,path fading= _htcz8osfa ,fading transform={xshift=2}] (281.6,76.28) .. controls (281.6,66.96) and (289.16,59.4) .. (298.48,59.4) -- (349.12,59.4) .. controls (358.44,59.4) and (366,66.96) .. (366,76.28) -- (366,274.52) .. controls (366,283.84) and (358.44,291.4) .. (349.12,291.4) -- (298.48,291.4) .. controls (289.16,291.4) and (281.6,283.84) .. (281.6,274.52) -- cycle ; % for fading 
     % \draw   (281.6,76.28) .. controls (281.6,66.96) and (289.16,59.4) .. (298.48,59.4) -- (349.12,59.4) .. controls (358.44,59.4) and (366,66.96) .. (366,76.28) -- (366,274.52) .. controls (366,283.84) and (358.44,291.4) .. (349.12,291.4) -- (298.48,291.4) .. controls (289.16,291.4) and (281.6,283.84) .. (281.6,274.52) -- cycle ; % for border 
    
    %Straight Lines [id:da12591274303854672] 
    \draw    (286,113.8) -- (296,113.8) ;
    %Straight Lines [id:da6889475605006761] 
    \draw    (306,113.8) -- (326,113.8) ;
    %Straight Lines [id:da6840209064516296] 
    \draw    (346,113.8) -- (366,113.8) ;
    %Straight Lines [id:da5182517300873626] 
    \draw    (286,143.8) -- (296,143.8) ;
    %Straight Lines [id:da6622021804539293] 
    \draw    (306,143.8) -- (326,143.8) ;
    %Straight Lines [id:da10986080077044069] 
    \draw    (346,143.8) -- (366,143.8) ;
    %Straight Lines [id:da06891079444490966] 
    \draw    (286,173.8) -- (296,173.8) ;
    %Straight Lines [id:da7088700201192246] 
    \draw    (306,173.8) -- (326,173.8) ;
    %Straight Lines [id:da8985551907012945] 
    \draw    (346,173.8) -- (366,173.8) ;
    %Straight Lines [id:da48183061562007] 
    \draw    (346,143.8) -- (366,143.8) ;
    %Straight Lines [id:da29259596379627184] 
    \draw    (346,173.8) -- (366,173.8) ;
    %Straight Lines [id:da10525896181304528] 
    \draw    (286,243.8) -- (296,243.8) ;
    %Straight Lines [id:da302504314759217] 
    \draw    (306,243.8) -- (326,243.8) ;
    %Straight Lines [id:da8403039577473146] 
    \draw    (346,243.8) -- (366,243.8) ;
    %Straight Lines [id:da5032229548045282] 
    \draw    (439,243.8) -- (449,243.8) ;
    %Straight Lines [id:da6606636042343228] 
    \draw    (439,173.8) -- (449,173.8) ;
    %Straight Lines [id:da648800320875722] 
    \draw    (439,143.8) -- (449,143.8) ;
    %Straight Lines [id:da3048783844469498] 
    \draw    (439,113.8) -- (449,113.8) ;
    %Shape: Circle [id:dp6177125527635312] 
    \draw  [fill={rgb, 255:red, 74; green, 144; blue, 226 }  ,fill opacity=1 ] (309,173.8) .. controls (309,172.14) and (310.34,170.8) .. (312,170.8) .. controls (313.66,170.8) and (315,172.14) .. (315,173.8) .. controls (315,175.46) and (313.66,176.8) .. (312,176.8) .. controls (310.34,176.8) and (309,175.46) .. (309,173.8) -- cycle ;
    %Shape: Circle [id:dp9641797724088836] 
    \draw  [fill={rgb, 255:red, 74; green, 144; blue, 226 }  ,fill opacity=1 ] (444,173.8) .. controls (444,172.14) and (445.34,170.8) .. (447,170.8) .. controls (448.66,170.8) and (450,172.14) .. (450,173.8) .. controls (450,175.46) and (448.66,176.8) .. (447,176.8) .. controls (445.34,176.8) and (444,175.46) .. (444,173.8) -- cycle ;
    %Shape: Circle [id:dp5216399345009763] 
    \draw  [fill={rgb, 255:red, 74; green, 144; blue, 226 }  ,fill opacity=1 ] (444,243.8) .. controls (444,242.14) and (445.34,240.8) .. (447,240.8) .. controls (448.66,240.8) and (450,242.14) .. (450,243.8) .. controls (450,245.46) and (448.66,246.8) .. (447,246.8) .. controls (445.34,246.8) and (444,245.46) .. (444,243.8) -- cycle ;
    %Shape: Circle [id:dp8730561346145524] 
    \draw  [fill={rgb, 255:red, 74; green, 144; blue, 226 }  ,fill opacity=1 ] (444,143.8) .. controls (444,142.14) and (445.34,140.8) .. (447,140.8) .. controls (448.66,140.8) and (450,142.14) .. (450,143.8) .. controls (450,145.46) and (448.66,146.8) .. (447,146.8) .. controls (445.34,146.8) and (444,145.46) .. (444,143.8) -- cycle ;
    %Shape: Circle [id:dp20793574265872938] 
    \draw  [fill={rgb, 255:red, 74; green, 144; blue, 226 }  ,fill opacity=1 ] (444,113.8) .. controls (444,112.14) and (445.34,110.8) .. (447,110.8) .. controls (448.66,110.8) and (450,112.14) .. (450,113.8) .. controls (450,115.46) and (448.66,116.8) .. (447,116.8) .. controls (445.34,116.8) and (444,115.46) .. (444,113.8) -- cycle ;
    %Straight Lines [id:da07747816733572055] 
    \draw    (271,113.8) -- (286,113.8) ;
    %Straight Lines [id:da46458200365321267] 
    \draw    (271,143.8) -- (286,143.8) ;
    %Straight Lines [id:da6911160749578646] 
    \draw    (271,173.8) -- (286,173.8) ;
    %Straight Lines [id:da977764753764023] 
    \draw    (271,243.8) -- (286,243.8) ;
    %Shape: Circle [id:dp2077543137374267] 
    \draw  [fill={rgb, 255:red, 74; green, 144; blue, 226 }  ,fill opacity=1 ] (268,113.8) .. controls (268,112.14) and (269.34,110.8) .. (271,110.8) .. controls (272.66,110.8) and (274,112.14) .. (274,113.8) .. controls (274,115.46) and (272.66,116.8) .. (271,116.8) .. controls (269.34,116.8) and (268,115.46) .. (268,113.8) -- cycle ;
    %Shape: Circle [id:dp21380722237130656] 
    \draw  [fill={rgb, 255:red, 74; green, 144; blue, 226 }  ,fill opacity=1 ] (268,143.8) .. controls (268,142.14) and (269.34,140.8) .. (271,140.8) .. controls (272.66,140.8) and (274,142.14) .. (274,143.8) .. controls (274,145.46) and (272.66,146.8) .. (271,146.8) .. controls (269.34,146.8) and (268,145.46) .. (268,143.8) -- cycle ;
    %Shape: Circle [id:dp1680590103325741] 
    \draw   (268,173.8) .. controls (268,172.14) and (269.34,170.8) .. (271,170.8) .. controls (272.66,170.8) and (274,172.14) .. (274,173.8) .. controls (274,175.46) and (272.66,176.8) .. (271,176.8) .. controls (269.34,176.8) and (268,175.46) .. (268,173.8) -- cycle ;
    %Shape: Circle [id:dp7473613782402011] 
    \draw  [fill={rgb, 255:red, 255; green, 255; blue, 255 }  ,fill opacity=1 ] (266.5,173.8) .. controls (266.5,171.31) and (268.51,169.3) .. (271,169.3) .. controls (273.49,169.3) and (275.5,171.31) .. (275.5,173.8) .. controls (275.5,176.29) and (273.49,178.3) .. (271,178.3) .. controls (268.51,178.3) and (266.5,176.29) .. (266.5,173.8) -- cycle ;
    %Curve Lines [id:da44907258759639634] 
    \draw    (278,173.8) .. controls (287.76,173.88) and (282.25,149.83) .. (296,149.8) ;
    %Curve Lines [id:da9408681678935567] 
    \draw    (274,273.8) .. controls (287,273.45) and (282.25,249.83) .. (296,249.8) ;
    %Curve Lines [id:da6386811890007127] 
    \draw    (296,197.8) .. controls (281.95,197.88) and (288.96,174.12) .. (278,173.8) ;
    %Curve Lines [id:da8138750209583685] 
    \draw    (296,107.8) .. controls (281.95,107.88) and (288,83.95) .. (274,83.8) ;
    %Straight Lines [id:da3653413273236392] 
    \draw    (325.5,113.8) -- (335.5,113.8) ;
    %Straight Lines [id:da16432948773608946] 
    \draw    (325.5,143.8) -- (335.5,143.8) ;
    %Straight Lines [id:da6739770002305319] 
    \draw    (325.5,173.8) -- (335.5,173.8) ;
    %Straight Lines [id:da2884180319037881] 
    \draw    (325.5,243.8) -- (335.5,243.8) ;
    %Straight Lines [id:da7251828149781272] 
    \draw    (315,113.8) -- (325.5,113.8) ;
    %Straight Lines [id:da09058450500132387] 
    \draw    (315,143.8) -- (325.5,143.8) ;
    %Straight Lines [id:da3921065083493407] 
    \draw    (315,173.8) -- (325.5,173.8) ;
    %Straight Lines [id:da007877760257336996] 
    \draw    (315,243.8) -- (325.5,243.8) ;
    %Straight Lines [id:da8374371156226426] 
    \draw    (366,113.8) -- (376,113.8) ;
    %Straight Lines [id:da9678309129770906] 
    \draw    (366,143.8) -- (376,143.8) ;
    %Straight Lines [id:da7361499806097593] 
    \draw    (366,173.8) -- (376,173.8) ;
    %Straight Lines [id:da6319078106662868] 
    \draw    (366,243.8) -- (376,243.8) ;
    %Straight Lines [id:da4405211719951492] 
    \draw    (355,113.8) -- (366,113.8) ;
    %Straight Lines [id:da4252229620947994] 
    \draw    (355,143.8) -- (366,143.8) ;
    %Straight Lines [id:da890591025300383] 
    \draw    (355,173.8) -- (366,173.8) ;
    %Straight Lines [id:da9131249668505879] 
    \draw    (355,243.8) -- (366,243.8) ;
    %Straight Lines [id:da7592375019488422] 
    \draw    (399.6,113.76) -- (419.6,113.76) ;
    %Straight Lines [id:da24129797069017322] 
    \draw    (399.6,143.76) -- (419.6,143.76) ;
    %Straight Lines [id:da9245052427082239] 
    \draw    (399.6,173.76) -- (419.6,173.76) ;
    %Straight Lines [id:da7011628293134007] 
    \draw    (399.6,243.76) -- (419.6,243.76) ;
    %Straight Lines [id:da8499202784094941] 
    \draw    (419.1,113.76) -- (429.1,113.76) ;
    %Straight Lines [id:da6204904262909317] 
    \draw    (419.1,143.76) -- (429.1,143.76) ;
    %Straight Lines [id:da03663650815216268] 
    \draw    (419.1,173.76) -- (429.1,173.76) ;
    %Straight Lines [id:da5227514591137388] 
    \draw    (419.1,243.76) -- (429.1,243.76) ;
    %Straight Lines [id:da38681589798229576] 
    \draw    (407.6,113.76) -- (419.1,113.76) ;
    %Straight Lines [id:da54264693532568] 
    \draw    (407.6,143.76) -- (419.1,143.76) ;
    %Straight Lines [id:da32474848386202826] 
    \draw    (407.6,173.76) -- (419.1,173.76) ;
    %Straight Lines [id:da8658223172801294] 
    \draw    (407.6,243.76) -- (419.1,243.76) ;
    %Shape: Circle [id:dp9427971243264817] 
    \draw  [fill={rgb, 255:red, 74; green, 144; blue, 226 }  ,fill opacity=1 ] (402,113.8) .. controls (402,112.14) and (403.34,110.8) .. (405,110.8) .. controls (406.66,110.8) and (408,112.14) .. (408,113.8) .. controls (408,115.46) and (406.66,116.8) .. (405,116.8) .. controls (403.34,116.8) and (402,115.46) .. (402,113.8) -- cycle ;
    %Shape: Circle [id:dp46086243960357365] 
    \draw  [fill={rgb, 255:red, 74; green, 144; blue, 226 }  ,fill opacity=1 ] (402,143.76) .. controls (402,142.1) and (403.34,140.76) .. (405,140.76) .. controls (406.66,140.76) and (408,142.1) .. (408,143.76) .. controls (408,145.42) and (406.66,146.76) .. (405,146.76) .. controls (403.34,146.76) and (402,145.42) .. (402,143.76) -- cycle ;
    %Shape: Circle [id:dp6620083068720557] 
    \draw  [fill={rgb, 255:red, 74; green, 144; blue, 226 }  ,fill opacity=1 ] (402,173.8) .. controls (402,172.14) and (403.34,170.8) .. (405,170.8) .. controls (406.66,170.8) and (408,172.14) .. (408,173.8) .. controls (408,175.46) and (406.66,176.8) .. (405,176.8) .. controls (403.34,176.8) and (402,175.46) .. (402,173.8) -- cycle ;
    %Shape: Circle [id:dp9202066916344684] 
    \draw  [fill={rgb, 255:red, 74; green, 144; blue, 226 }  ,fill opacity=1 ] (402,243.8) .. controls (402,242.14) and (403.34,240.8) .. (405,240.8) .. controls (406.66,240.8) and (408,242.14) .. (408,243.8) .. controls (408,245.46) and (406.66,246.8) .. (405,246.8) .. controls (403.34,246.8) and (402,245.46) .. (402,243.8) -- cycle ;
    %Curve Lines [id:da32334900655572585] 
    \draw    (314,273.8) .. controls (327,273.45) and (322.25,249.83) .. (336,249.8) ;
    %Curve Lines [id:da11724765940385651] 
    \draw    (354,273.8) .. controls (367,273.45) and (362.25,249.83) .. (376,249.8) ;
    %Curve Lines [id:da7982105615434423] 
    \draw    (407.1,273.76) .. controls (420.1,273.41) and (415.35,249.79) .. (429.1,249.76) ;
    %Curve Lines [id:da004908421581965694] 
    \draw    (336,107.8) .. controls (321.95,107.88) and (328,83.95) .. (314,83.8) ;
    %Curve Lines [id:da9511690430401551] 
    \draw    (376,107.8) .. controls (361.95,107.88) and (368,83.95) .. (354,83.8) ;
    %Curve Lines [id:da9914997890244512] 
    \draw    (429.1,107.76) .. controls (415.05,107.84) and (421.1,83.91) .. (407.1,83.76) ;
    %Rounded Rect [id:dp42702965624068967] 
    \draw  [fill={rgb, 255:red, 235; green, 83; blue, 112 }  ,fill opacity=1 ] (293.67,104.2) .. controls (293.67,102.87) and (294.74,101.8) .. (296.07,101.8) -- (303.27,101.8) .. controls (304.59,101.8) and (305.67,102.87) .. (305.67,104.2) -- (305.67,123.4) .. controls (305.67,124.73) and (304.59,125.8) .. (303.27,125.8) -- (296.07,125.8) .. controls (294.74,125.8) and (293.67,124.73) .. (293.67,123.4) -- cycle ;
    %Curve Lines [id:da06858342935056427] 
    \draw    (278,143.8) .. controls (287.76,143.88) and (280.18,119.86) .. (293.93,119.83) ;
    %Curve Lines [id:da723298045787133] 
    \draw    (317.5,143.8) .. controls (327.26,143.88) and (321.75,119.83) .. (335.5,119.8) ;
    %Curve Lines [id:da2512187850282652] 
    \draw    (358,143.8) .. controls (367.76,143.88) and (362.25,119.83) .. (376,119.8) ;
    %Curve Lines [id:da9053084993063347] 
    \draw    (411.1,143.76) .. controls (420.86,143.84) and (415.35,119.79) .. (429.1,119.76) ;
    %Curve Lines [id:da92353097157552] 
    \draw    (317.5,173.8) .. controls (327.26,173.88) and (321.75,149.83) .. (335.5,149.8) ;
    %Curve Lines [id:da24776318146114673] 
    \draw    (358,173.8) .. controls (367.76,173.88) and (362.25,149.83) .. (376,149.8) ;
    %Curve Lines [id:da07442995908116457] 
    \draw    (411.1,173.76) .. controls (420.86,173.84) and (415.35,149.79) .. (429.1,149.76) ;
    %Curve Lines [id:da956818800686466] 
    \draw    (278,203.8) .. controls (287.76,203.88) and (282.25,179.83) .. (296,179.8) ;
    %Curve Lines [id:da06594681326228691] 
    \draw    (317.5,203.8) .. controls (327.26,203.88) and (321.75,179.83) .. (335.5,179.8) ;
    %Curve Lines [id:da5916140616891901] 
    \draw    (411.1,203.76) .. controls (420.86,203.84) and (415.35,179.79) .. (429.1,179.76) ;
    %Curve Lines [id:da06825346822128453] 
    \draw    (296,137.8) .. controls (281.95,137.88) and (288.96,114.12) .. (278,113.8) ;
    %Curve Lines [id:da5729440546733633] 
    \draw    (296,167.8) .. controls (281.95,167.88) and (288.96,144.12) .. (278,143.8) ;
    %Curve Lines [id:da2689295741977933] 
    \draw    (335.5,137.8) .. controls (321.45,137.88) and (328.46,114.12) .. (317.5,113.8) ;
    %Curve Lines [id:da8736063287654374] 
    \draw    (335.5,167.8) .. controls (321.45,167.88) and (328.46,144.12) .. (317.5,143.8) ;
    %Curve Lines [id:da315584754437392] 
    \draw    (335.5,197.8) .. controls (321.45,197.88) and (328.46,174.12) .. (317.5,173.8) ;
    %Curve Lines [id:da46919548113502985] 
    \draw    (376,197.8) .. controls (361.95,197.88) and (368.96,174.12) .. (358,173.8) ;
    %Curve Lines [id:da5791093989238294] 
    \draw    (376,167.8) .. controls (361.95,167.88) and (368.96,144.12) .. (358,143.8) ;
    %Curve Lines [id:da2995343957801271] 
    \draw    (376,137.8) .. controls (361.95,137.88) and (368.96,114.12) .. (358,113.8) ;
    %Curve Lines [id:da6279904209902487] 
    \draw    (429.1,137.76) .. controls (415.05,137.84) and (422.06,114.08) .. (411.1,113.76) ;
    %Curve Lines [id:da7566681121693786] 
    \draw    (429.1,167.76) .. controls (415.05,167.84) and (422.06,144.08) .. (411.1,143.76) ;
    %Curve Lines [id:da7687108096378197] 
    \draw    (429.1,197.76) .. controls (415.05,197.84) and (422.06,174.08) .. (411.1,173.76) ;
    %Curve Lines [id:da9906758036883361] 
    \draw    (278,243.8) .. controls (287.76,243.88) and (282.25,219.83) .. (296,219.8) ;
    %Curve Lines [id:da6426692743902085] 
    \draw    (317.5,243.8) .. controls (327.26,243.88) and (321.75,219.83) .. (335.5,219.8) ;
    %Curve Lines [id:da7324254375871793] 
    \draw    (358,243.8) .. controls (367.76,243.88) and (362.25,219.83) .. (376,219.8) ;
    %Curve Lines [id:da03023041445865271] 
    \draw    (411.1,243.76) .. controls (420.86,243.84) and (415.35,219.79) .. (429.1,219.76) ;
    %Curve Lines [id:da6837151140111912] 
    \draw    (296,237.8) .. controls (281.95,237.88) and (288.96,214.12) .. (278,213.8) ;
    %Curve Lines [id:da7387383084996648] 
    \draw    (335.5,237.8) .. controls (321.45,237.88) and (328.46,214.12) .. (317.5,213.8) ;
    %Curve Lines [id:da9210010738401084] 
    \draw    (429.1,237.76) .. controls (415.05,237.84) and (422.06,214.08) .. (411.1,213.76) ;
    %Shape: Circle [id:dp20146326966731642] 
    \draw  [fill={rgb, 255:red, 255; green, 255; blue, 255 }  ,fill opacity=1 ] (308.5,173.8) .. controls (308.5,171.31) and (310.51,169.3) .. (313,169.3) .. controls (315.49,169.3) and (317.5,171.31) .. (317.5,173.8) .. controls (317.5,176.29) and (315.49,178.3) .. (313,178.3) .. controls (310.51,178.3) and (308.5,176.29) .. (308.5,173.8) -- cycle ;
    %Shape: Circle [id:dp4487595160974982] 
    \draw  [fill={rgb, 255:red, 255; green, 255; blue, 255 }  ,fill opacity=1 ] (266.5,143.8) .. controls (266.5,141.31) and (268.51,139.3) .. (271,139.3) .. controls (273.49,139.3) and (275.5,141.31) .. (275.5,143.8) .. controls (275.5,146.29) and (273.49,148.3) .. (271,148.3) .. controls (268.51,148.3) and (266.5,146.29) .. (266.5,143.8) -- cycle ;
    %Shape: Circle [id:dp9195781546216495] 
    \draw  [fill={rgb, 255:red, 255; green, 255; blue, 255 }  ,fill opacity=1 ] (266.5,113.8) .. controls (266.5,111.31) and (268.51,109.3) .. (271,109.3) .. controls (273.49,109.3) and (275.5,111.31) .. (275.5,113.8) .. controls (275.5,116.29) and (273.49,118.3) .. (271,118.3) .. controls (268.51,118.3) and (266.5,116.29) .. (266.5,113.8) -- cycle ;
    %Shape: Circle [id:dp6116837286965402] 
    \draw  [fill={rgb, 255:red, 255; green, 255; blue, 255 }  ,fill opacity=1 ] (269.5,83.8) .. controls (269.5,81.31) and (271.51,79.3) .. (274,79.3) .. controls (276.49,79.3) and (278.5,81.31) .. (278.5,83.8) .. controls (278.5,86.29) and (276.49,88.3) .. (274,88.3) .. controls (271.51,88.3) and (269.5,86.29) .. (269.5,83.8) -- cycle ;
    %Shape: Circle [id:dp3625678025113275] 
    \draw  [fill={rgb, 255:red, 255; green, 255; blue, 255 }  ,fill opacity=1 ] (308.5,143.8) .. controls (308.5,141.31) and (310.51,139.3) .. (313,139.3) .. controls (315.49,139.3) and (317.5,141.31) .. (317.5,143.8) .. controls (317.5,146.29) and (315.49,148.3) .. (313,148.3) .. controls (310.51,148.3) and (308.5,146.29) .. (308.5,143.8) -- cycle ;
    %Shape: Circle [id:dp03312322880961438] 
    \draw  [fill={rgb, 255:red, 255; green, 255; blue, 255 }  ,fill opacity=1 ] (308.5,113.8) .. controls (308.5,111.31) and (310.51,109.3) .. (313,109.3) .. controls (315.49,109.3) and (317.5,111.31) .. (317.5,113.8) .. controls (317.5,116.29) and (315.49,118.3) .. (313,118.3) .. controls (310.51,118.3) and (308.5,116.29) .. (308.5,113.8) -- cycle ;
    %Shape: Circle [id:dp043865844405652554] 
    \draw  [fill={rgb, 255:red, 255; green, 255; blue, 255 }  ,fill opacity=1 ] (309.5,83.8) .. controls (309.5,81.31) and (311.51,79.3) .. (314,79.3) .. controls (316.49,79.3) and (318.5,81.31) .. (318.5,83.8) .. controls (318.5,86.29) and (316.49,88.3) .. (314,88.3) .. controls (311.51,88.3) and (309.5,86.29) .. (309.5,83.8) -- cycle ;
    %Shape: Circle [id:dp21578576624907608] 
    \draw  [fill={rgb, 255:red, 255; green, 255; blue, 255 }  ,fill opacity=1 ] (349.5,83.8) .. controls (349.5,81.31) and (351.51,79.3) .. (354,79.3) .. controls (356.49,79.3) and (358.5,81.31) .. (358.5,83.8) .. controls (358.5,86.29) and (356.49,88.3) .. (354,88.3) .. controls (351.51,88.3) and (349.5,86.29) .. (349.5,83.8) -- cycle ;
    %Shape: Circle [id:dp28073888823911486] 
    \draw  [fill={rgb, 255:red, 255; green, 255; blue, 255 }  ,fill opacity=1 ] (349,113.8) .. controls (349,111.31) and (351.01,109.3) .. (353.5,109.3) .. controls (355.99,109.3) and (358,111.31) .. (358,113.8) .. controls (358,116.29) and (355.99,118.3) .. (353.5,118.3) .. controls (351.01,118.3) and (349,116.29) .. (349,113.8) -- cycle ;
    %Shape: Circle [id:dp5081744133289241] 
    \draw  [fill={rgb, 255:red, 255; green, 255; blue, 255 }  ,fill opacity=1 ] (349,143.8) .. controls (349,141.31) and (351.01,139.3) .. (353.5,139.3) .. controls (355.99,139.3) and (358,141.31) .. (358,143.8) .. controls (358,146.29) and (355.99,148.3) .. (353.5,148.3) .. controls (351.01,148.3) and (349,146.29) .. (349,143.8) -- cycle ;
    %Shape: Circle [id:dp01748356301794951] 
    \draw  [fill={rgb, 255:red, 255; green, 255; blue, 255 }  ,fill opacity=1 ] (349,173.8) .. controls (349,171.31) and (351.01,169.3) .. (353.5,169.3) .. controls (355.99,169.3) and (358,171.31) .. (358,173.8) .. controls (358,176.29) and (355.99,178.3) .. (353.5,178.3) .. controls (351.01,178.3) and (349,176.29) .. (349,173.8) -- cycle ;
    %Shape: Circle [id:dp2526359291560343] 
    \draw  [fill={rgb, 255:red, 255; green, 255; blue, 255 }  ,fill opacity=1 ] (266.5,243.8) .. controls (266.5,241.31) and (268.51,239.3) .. (271,239.3) .. controls (273.49,239.3) and (275.5,241.31) .. (275.5,243.8) .. controls (275.5,246.29) and (273.49,248.3) .. (271,248.3) .. controls (268.51,248.3) and (266.5,246.29) .. (266.5,243.8) -- cycle ;
    %Shape: Circle [id:dp30795312394062346] 
    \draw  [fill={rgb, 255:red, 255; green, 255; blue, 255 }  ,fill opacity=1 ] (269.5,273.8) .. controls (269.5,271.31) and (271.51,269.3) .. (274,269.3) .. controls (276.49,269.3) and (278.5,271.31) .. (278.5,273.8) .. controls (278.5,276.29) and (276.49,278.3) .. (274,278.3) .. controls (271.51,278.3) and (269.5,276.29) .. (269.5,273.8) -- cycle ;
    %Shape: Circle [id:dp6531679226550287] 
    \draw  [fill={rgb, 255:red, 255; green, 255; blue, 255 }  ,fill opacity=1 ] (309.5,273.8) .. controls (309.5,271.31) and (311.51,269.3) .. (314,269.3) .. controls (316.49,269.3) and (318.5,271.31) .. (318.5,273.8) .. controls (318.5,276.29) and (316.49,278.3) .. (314,278.3) .. controls (311.51,278.3) and (309.5,276.29) .. (309.5,273.8) -- cycle ;
    %Shape: Circle [id:dp23741910846556058] 
    \draw  [fill={rgb, 255:red, 255; green, 255; blue, 255 }  ,fill opacity=1 ] (349.5,273.8) .. controls (349.5,271.31) and (351.51,269.3) .. (354,269.3) .. controls (356.49,269.3) and (358.5,271.31) .. (358.5,273.8) .. controls (358.5,276.29) and (356.49,278.3) .. (354,278.3) .. controls (351.51,278.3) and (349.5,276.29) .. (349.5,273.8) -- cycle ;
    %Shape: Circle [id:dp8799962376767778] 
    \draw  [fill={rgb, 255:red, 255; green, 255; blue, 255 }  ,fill opacity=1 ] (308.5,243.8) .. controls (308.5,241.31) and (310.51,239.3) .. (313,239.3) .. controls (315.49,239.3) and (317.5,241.31) .. (317.5,243.8) .. controls (317.5,246.29) and (315.49,248.3) .. (313,248.3) .. controls (310.51,248.3) and (308.5,246.29) .. (308.5,243.8) -- cycle ;
    %Shape: Circle [id:dp5436648552880832] 
    \draw  [fill={rgb, 255:red, 255; green, 255; blue, 255 }  ,fill opacity=1 ] (349,243.8) .. controls (349,241.31) and (351.01,239.3) .. (353.5,239.3) .. controls (355.99,239.3) and (358,241.31) .. (358,243.8) .. controls (358,246.29) and (355.99,248.3) .. (353.5,248.3) .. controls (351.01,248.3) and (349,246.29) .. (349,243.8) -- cycle ;
    %Shape: Circle [id:dp4508122471949372] 
    \draw  [fill={rgb, 255:red, 255; green, 255; blue, 255 }  ,fill opacity=1 ] (402.6,83.76) .. controls (402.6,81.27) and (404.61,79.26) .. (407.1,79.26) .. controls (409.59,79.26) and (411.6,81.27) .. (411.6,83.76) .. controls (411.6,86.25) and (409.59,88.26) .. (407.1,88.26) .. controls (404.61,88.26) and (402.6,86.25) .. (402.6,83.76) -- cycle ;
    %Shape: Circle [id:dp7623420038386579] 
    \draw  [fill={rgb, 255:red, 255; green, 255; blue, 255 }  ,fill opacity=1 ] (402,113.8) .. controls (402,111.31) and (404.01,109.3) .. (406.5,109.3) .. controls (408.99,109.3) and (411,111.31) .. (411,113.8) .. controls (411,116.29) and (408.99,118.3) .. (406.5,118.3) .. controls (404.01,118.3) and (402,116.29) .. (402,113.8) -- cycle ;
    %Shape: Circle [id:dp6338949287421527] 
    \draw  [fill={rgb, 255:red, 255; green, 255; blue, 255 }  ,fill opacity=1 ] (402.1,143.76) .. controls (402.1,141.27) and (404.11,139.26) .. (406.6,139.26) .. controls (409.09,139.26) and (411.1,141.27) .. (411.1,143.76) .. controls (411.1,146.25) and (409.09,148.26) .. (406.6,148.26) .. controls (404.11,148.26) and (402.1,146.25) .. (402.1,143.76) -- cycle ;
    %Shape: Circle [id:dp9494208779127] 
    \draw  [fill={rgb, 255:red, 255; green, 255; blue, 255 }  ,fill opacity=1 ] (402,173.8) .. controls (402,171.31) and (404.01,169.3) .. (406.5,169.3) .. controls (408.99,169.3) and (411,171.31) .. (411,173.8) .. controls (411,176.29) and (408.99,178.3) .. (406.5,178.3) .. controls (404.01,178.3) and (402,176.29) .. (402,173.8) -- cycle ;
    %Shape: Circle [id:dp7362804325370593] 
    \draw  [fill={rgb, 255:red, 255; green, 255; blue, 255 }  ,fill opacity=1 ] (402,243.8) .. controls (402,241.31) and (404.01,239.3) .. (406.5,239.3) .. controls (408.99,239.3) and (411,241.31) .. (411,243.8) .. controls (411,246.29) and (408.99,248.3) .. (406.5,248.3) .. controls (404.01,248.3) and (402,246.29) .. (402,243.8) -- cycle ;
    %Shape: Circle [id:dp9761672870224487] 
    \draw  [fill={rgb, 255:red, 255; green, 255; blue, 255 }  ,fill opacity=1 ] (402.6,273.76) .. controls (402.6,271.27) and (404.61,269.26) .. (407.1,269.26) .. controls (409.59,269.26) and (411.6,271.27) .. (411.6,273.76) .. controls (411.6,276.25) and (409.59,278.26) .. (407.1,278.26) .. controls (404.61,278.26) and (402.6,276.25) .. (402.6,273.76) -- cycle ;
    %Shape: Circle [id:dp3587618015248244] 
    \draw  [fill={rgb, 255:red, 255; green, 255; blue, 255 }  ,fill opacity=1 ] (444,113.8) .. controls (444,111.31) and (446.01,109.3) .. (448.5,109.3) .. controls (450.99,109.3) and (453,111.31) .. (453,113.8) .. controls (453,116.29) and (450.99,118.3) .. (448.5,118.3) .. controls (446.01,118.3) and (444,116.29) .. (444,113.8) -- cycle ;
    %Shape: Circle [id:dp257482515582123] 
    \draw  [fill={rgb, 255:red, 255; green, 255; blue, 255 }  ,fill opacity=1 ] (444,143.8) .. controls (444,141.31) and (446.01,139.3) .. (448.5,139.3) .. controls (450.99,139.3) and (453,141.31) .. (453,143.8) .. controls (453,146.29) and (450.99,148.3) .. (448.5,148.3) .. controls (446.01,148.3) and (444,146.29) .. (444,143.8) -- cycle ;
    %Shape: Circle [id:dp8680615145799881] 
    \draw  [fill={rgb, 255:red, 255; green, 255; blue, 255 }  ,fill opacity=1 ] (444,173.8) .. controls (444,171.31) and (446.01,169.3) .. (448.5,169.3) .. controls (450.99,169.3) and (453,171.31) .. (453,173.8) .. controls (453,176.29) and (450.99,178.3) .. (448.5,178.3) .. controls (446.01,178.3) and (444,176.29) .. (444,173.8) -- cycle ;
    %Shape: Circle [id:dp4306579762720619] 
    \draw  [fill={rgb, 255:red, 255; green, 255; blue, 255 }  ,fill opacity=1 ] (444,243.8) .. controls (444,241.31) and (446.01,239.3) .. (448.5,239.3) .. controls (450.99,239.3) and (453,241.31) .. (453,243.8) .. controls (453,246.29) and (450.99,248.3) .. (448.5,248.3) .. controls (446.01,248.3) and (444,246.29) .. (444,243.8) -- cycle ;
    %Rounded Rect [id:dp9521886425866692] 
    \draw  [fill={rgb, 255:red, 235; green, 83; blue, 112 }  ,fill opacity=1 ] (293.67,134.2) .. controls (293.67,132.87) and (294.74,131.8) .. (296.07,131.8) -- (303.27,131.8) .. controls (304.59,131.8) and (305.67,132.87) .. (305.67,134.2) -- (305.67,153.4) .. controls (305.67,154.73) and (304.59,155.8) .. (303.27,155.8) -- (296.07,155.8) .. controls (294.74,155.8) and (293.67,154.73) .. (293.67,153.4) -- cycle ;
    %Rounded Rect [id:dp9007360317843287] 
    \draw  [fill={rgb, 255:red, 235; green, 83; blue, 112 }  ,fill opacity=1 ] (293.7,164.2) .. controls (293.7,162.87) and (294.77,161.8) .. (296.1,161.8) -- (303.3,161.8) .. controls (304.63,161.8) and (305.7,162.87) .. (305.7,164.2) -- (305.7,183.4) .. controls (305.7,184.73) and (304.63,185.8) .. (303.3,185.8) -- (296.1,185.8) .. controls (294.77,185.8) and (293.7,184.73) .. (293.7,183.4) -- cycle ;
    %Rounded Rect [id:dp7755940946979027] 
    \draw  [fill={rgb, 255:red, 235; green, 83; blue, 112 }  ,fill opacity=1 ] (293.7,233.87) .. controls (293.7,232.54) and (294.77,231.47) .. (296.1,231.47) -- (303.3,231.47) .. controls (304.63,231.47) and (305.7,232.54) .. (305.7,233.87) -- (305.7,253.07) .. controls (305.7,254.39) and (304.63,255.47) .. (303.3,255.47) -- (296.1,255.47) .. controls (294.77,255.47) and (293.7,254.39) .. (293.7,253.07) -- cycle ;
    %Rounded Rect [id:dp3770236077825324] 
    \draw  [fill={rgb, 255:red, 235; green, 83; blue, 112 }  ,fill opacity=1 ] (333.67,104.2) .. controls (333.67,102.87) and (334.74,101.8) .. (336.07,101.8) -- (343.27,101.8) .. controls (344.59,101.8) and (345.67,102.87) .. (345.67,104.2) -- (345.67,123.4) .. controls (345.67,124.73) and (344.59,125.8) .. (343.27,125.8) -- (336.07,125.8) .. controls (334.74,125.8) and (333.67,124.73) .. (333.67,123.4) -- cycle ;
    %Rounded Rect [id:dp2955229197086444] 
    \draw  [fill={rgb, 255:red, 235; green, 83; blue, 112 }  ,fill opacity=1 ] (333.67,134.2) .. controls (333.67,132.87) and (334.74,131.8) .. (336.07,131.8) -- (343.27,131.8) .. controls (344.59,131.8) and (345.67,132.87) .. (345.67,134.2) -- (345.67,153.4) .. controls (345.67,154.73) and (344.59,155.8) .. (343.27,155.8) -- (336.07,155.8) .. controls (334.74,155.8) and (333.67,154.73) .. (333.67,153.4) -- cycle ;
    %Rounded Rect [id:dp3257551753328596] 
    \draw  [fill={rgb, 255:red, 235; green, 83; blue, 112 }  ,fill opacity=1 ] (333.7,164.2) .. controls (333.7,162.87) and (334.77,161.8) .. (336.1,161.8) -- (343.3,161.8) .. controls (344.63,161.8) and (345.7,162.87) .. (345.7,164.2) -- (345.7,183.4) .. controls (345.7,184.73) and (344.63,185.8) .. (343.3,185.8) -- (336.1,185.8) .. controls (334.77,185.8) and (333.7,184.73) .. (333.7,183.4) -- cycle ;
    %Rounded Rect [id:dp5015753026325627] 
    \draw  [fill={rgb, 255:red, 235; green, 83; blue, 112 }  ,fill opacity=1 ] (333.7,233.57) .. controls (333.7,232.24) and (334.77,231.17) .. (336.1,231.17) -- (343.3,231.17) .. controls (344.63,231.17) and (345.7,232.24) .. (345.7,233.57) -- (345.7,252.77) .. controls (345.7,254.09) and (344.63,255.17) .. (343.3,255.17) -- (336.1,255.17) .. controls (334.77,255.17) and (333.7,254.09) .. (333.7,252.77) -- cycle ;
    %Rounded Rect [id:dp0888424840477875] 
    \draw  [fill={rgb, 255:red, 235; green, 83; blue, 112 }  ,fill opacity=1 ] (427.7,103.87) .. controls (427.7,102.54) and (428.77,101.47) .. (430.1,101.47) -- (437.3,101.47) .. controls (438.63,101.47) and (439.7,102.54) .. (439.7,103.87) -- (439.7,123.07) .. controls (439.7,124.39) and (438.63,125.47) .. (437.3,125.47) -- (430.1,125.47) .. controls (428.77,125.47) and (427.7,124.39) .. (427.7,123.07) -- cycle ;
    %Rounded Rect [id:dp7785611532578929] 
    \draw  [fill={rgb, 255:red, 235; green, 83; blue, 112 }  ,fill opacity=1 ] (427.7,134.2) .. controls (427.7,132.87) and (428.77,131.8) .. (430.1,131.8) -- (437.3,131.8) .. controls (438.63,131.8) and (439.7,132.87) .. (439.7,134.2) -- (439.7,153.4) .. controls (439.7,154.73) and (438.63,155.8) .. (437.3,155.8) -- (430.1,155.8) .. controls (428.77,155.8) and (427.7,154.73) .. (427.7,153.4) -- cycle ;
    %Rounded Rect [id:dp7907921203888848] 
    \draw  [fill={rgb, 255:red, 235; green, 83; blue, 112 }  ,fill opacity=1 ] (427.73,164.2) .. controls (427.73,162.87) and (428.81,161.8) .. (430.13,161.8) -- (437.33,161.8) .. controls (438.66,161.8) and (439.73,162.87) .. (439.73,164.2) -- (439.73,183.4) .. controls (439.73,184.73) and (438.66,185.8) .. (437.33,185.8) -- (430.13,185.8) .. controls (428.81,185.8) and (427.73,184.73) .. (427.73,183.4) -- cycle ;
    %Rounded Rect [id:dp8701852355059259] 
    \draw  [fill={rgb, 255:red, 235; green, 83; blue, 112 }  ,fill opacity=1 ] (427.73,233.9) .. controls (427.73,232.57) and (428.81,231.5) .. (430.13,231.5) -- (437.33,231.5) .. controls (438.66,231.5) and (439.73,232.57) .. (439.73,233.9) -- (439.73,253.1) .. controls (439.73,254.43) and (438.66,255.5) .. (437.33,255.5) -- (430.13,255.5) .. controls (428.81,255.5) and (427.73,254.43) .. (427.73,253.1) -- cycle ;
    

%Straight Lines [id:da7675709313550161] 
\draw [dashed]   (424.25,128.9) -- (424.25,158.9) ;
%Straight Lines [id:da5513222139736342] 
\draw [dashed]   (424.25,128.9) -- (443.15,128.9) ;
%Straight Lines [id:da5724218915400858] 
% \draw [dashed]   (448.31,132.74) -- (472.65,132.7) ;
\draw[->][line width=1.0pt] (468,170.7) -- (446,154.7) ;
%Straight Lines [id:da98028289030947] 
\draw  [dashed]  (424.25,158.9) -- (443.15,158.85) ;
%Straight Lines [id:da3690262092301855] 
\draw  [dashed]  (443.15,128.9) -- (443.15,158.85) ;

%Shape: Circle [id:dp3980815479811872] 
\draw  [fill={rgb, 255:red, 255; green, 255; blue, 255 }  ,fill opacity=1 ] (491,154.1) .. controls (491,151.61) and (493.01,149.6) .. (495.5,149.6) .. controls (497.99,149.6) and (500,151.61) .. (500,154.1) .. controls (500,156.59) and (497.99,158.6) .. (495.5,158.6) .. controls (493.01,158.6) and (491,156.59) .. (491,154.1) -- cycle ;
%Shape: Circle [id:dp8208493476316323] 
\draw  [fill={rgb, 255:red, 255; green, 255; blue, 255 }  ,fill opacity=1 ] (491,191.1) .. controls (491,188.61) and (493.01,186.6) .. (495.5,186.6) .. controls (497.99,186.6) and (500,188.61) .. (500,191.1) .. controls (500,193.59) and (497.99,195.6) .. (495.5,195.6) .. controls (493.01,195.6) and (491,193.59) .. (491,191.1) -- cycle ;
%Shape: Circle [id:dp3173494411067954] 
\draw  [fill={rgb, 255:red, 255; green, 255; blue, 255 }  ,fill opacity=1 ] (478.67,172.6) .. controls (478.67,170.11) and (480.68,168.1) .. (483.17,168.1) .. controls (485.65,168.1) and (487.67,170.11) .. (487.67,172.6) .. controls (487.67,175.09) and (485.65,177.1) .. (483.17,177.1) .. controls (480.68,177.1) and (478.67,175.09) .. (478.67,172.6) -- cycle ;
%Shape: Circle [id:dp967421569400009] 
\draw  [fill={rgb, 255:red, 255; green, 255; blue, 255 }  ,fill opacity=1 ] (503.4,172.6) .. controls (503.4,170.11) and (505.41,168.1) .. (507.9,168.1) .. controls (510.39,168.1) and (512.4,170.11) .. (512.4,172.6) .. controls (512.4,175.09) and (510.39,177.1) .. (507.9,177.1) .. controls (505.41,177.1) and (503.4,175.09) .. (503.4,172.6) -- cycle ;
%Shape: Circle [id:dp938870324276512] 
\draw  [fill={rgb, 255:red, 255; green, 255; blue, 255 }  ,fill opacity=1 ] (524.7,172.5) .. controls (524.7,170.01) and (526.71,168) .. (529.2,168) .. controls (531.69,168) and (533.7,170.01) .. (533.7,172.5) .. controls (533.7,174.99) and (531.69,177) .. (529.2,177) .. controls (526.71,177) and (524.7,174.99) .. (524.7,172.5) -- cycle ;
%Rounded Rect [id:dp40904235215635065] 
\draw  [fill={rgb, 255:red, 235; green, 83; blue, 112 }  ,fill opacity=1 ] (475.6,155.04) .. controls (475.6,148.59) and (480.83,143.36) .. (487.28,143.36) -- (525.12,143.36) .. controls (531.57,143.36) and (536.8,148.59) .. (536.8,155.04) -- (536.8,190.08) .. controls (536.8,196.53) and (531.57,201.76) .. (525.12,201.76) -- (487.28,201.76) .. controls (480.83,201.76) and (475.6,196.53) .. (475.6,190.08) -- cycle ;
%Straight Lines [id:da6938220604961829] 
\draw    (491,154.1) -- (469.8,154.1) ;
%Straight Lines [id:da05667244881651223] 
\draw    (478.67,172.47) -- (469.8,172.5) ;
%Straight Lines [id:da7336130308225484] 
\draw    (491,191.1) -- (469.8,191.1) ;
%Straight Lines [id:da9632416883191204] 
\draw    (483.17,172.6) -- (495.5,154.1) ;
%Straight Lines [id:da8062551247029028] 
\draw    (483.17,172.6) -- (495.5,191.1) ;
%Straight Lines [id:da6815111207533529] 
\draw    (495.5,154.1) -- (507.9,172.6) ;
%Straight Lines [id:da2583696463055196] 
\draw    (495.5,191.1) -- (507.9,172.6) ;
%Straight Lines [id:da5112788448231447] 
\draw    (483.17,172.6) -- (507.9,172.6) ;
%Straight Lines [id:da21092112099851623] 
\draw    (495.5,191.1) -- (495.5,154.1) ;
%Curve Lines [id:da6381556136782309] 
\draw    (495.5,191.1) .. controls (509.12,191.04) and (521.92,189.04) .. (529.2,172.5) ;
%Curve Lines [id:da40382455887046964] 
\draw    (495.5,154.1) .. controls (511.92,153.84) and (524.32,158.24) .. (529.2,172.5) ;
%Curve Lines [id:da518536427411914] 
\draw    (529.2,172.5) .. controls (514.32,159.44) and (495.92,161.04) .. (483.17,172.6) ;
%Curve Lines [id:da5213323072929468] 
\draw    (529.2,172.5) .. controls (523.52,180.64) and (511.92,179.84) .. (507.9,172.6) ;
%Straight Lines [id:da26081701800713075] 
\draw    (533.7,172.5) -- (541.8,172.5) ;
%Shape: Circle [id:dp28592847133985866] 
\draw  [fill={rgb, 255:red, 255; green, 255; blue, 255 }  ,fill opacity=1 ] (478.67,172.6) .. controls (478.67,170.11) and (480.68,168.1) .. (483.17,168.1) .. controls (485.65,168.1) and (487.67,170.11) .. (487.67,172.6) .. controls (487.67,175.09) and (485.65,177.1) .. (483.17,177.1) .. controls (480.68,177.1) and (478.67,175.09) .. (478.67,172.6) -- cycle ;
%Shape: Circle [id:dp7005363351052425] 
\draw  [fill={rgb, 255:red, 255; green, 255; blue, 255 }  ,fill opacity=1 ] (491,154.1) .. controls (491,151.61) and (493.01,149.6) .. (495.5,149.6) .. controls (497.99,149.6) and (500,151.61) .. (500,154.1) .. controls (500,156.59) and (497.99,158.6) .. (495.5,158.6) .. controls (493.01,158.6) and (491,156.59) .. (491,154.1) -- cycle ;
%Shape: Circle [id:dp5131516311175763] 
\draw  [fill={rgb, 255:red, 255; green, 255; blue, 255 }  ,fill opacity=1 ] (491,191.1) .. controls (491,188.61) and (493.01,186.6) .. (495.5,186.6) .. controls (497.99,186.6) and (500,188.61) .. (500,191.1) .. controls (500,193.59) and (497.99,195.6) .. (495.5,195.6) .. controls (493.01,195.6) and (491,193.59) .. (491,191.1) -- cycle ;
%Shape: Circle [id:dp38829310474408074] 
\draw  [fill={rgb, 255:red, 255; green, 255; blue, 255 }  ,fill opacity=1 ] (503.4,172.6) .. controls (503.4,170.11) and (505.41,168.1) .. (507.9,168.1) .. controls (510.39,168.1) and (512.4,170.11) .. (512.4,172.6) .. controls (512.4,175.09) and (510.39,177.1) .. (507.9,177.1) .. controls (505.41,177.1) and (503.4,175.09) .. (503.4,172.6) -- cycle ;
%Shape: Circle [id:dp6003377245628321] 
\draw  [fill={rgb, 255:red, 255; green, 255; blue, 255 }  ,fill opacity=1 ] (524.7,172.5) .. controls (524.7,170.01) and (526.71,168) .. (529.2,168) .. controls (531.69,168) and (533.7,170.01) .. (533.7,172.5) .. controls (533.7,174.99) and (531.69,177) .. (529.2,177) .. controls (526.71,177) and (524.7,174.99) .. (524.7,172.5) -- cycle ;


    % Text Node
    \draw (379,111.8) node [anchor=north west][inner sep=0.75pt]   [align=left] {$\displaystyle \cdots $};
    % Text Node
    \draw (379,141.8) node [anchor=north west][inner sep=0.75pt]   [align=left] {$\displaystyle \cdots $};
    % Text Node
    \draw (379,172.09) node [anchor=north west][inner sep=0.75pt]   [align=left] {$\displaystyle \cdots $};
    % Text Node
    \draw (379,241.8) node [anchor=north west][inner sep=0.75pt]   [align=left] {$\displaystyle \cdots $};
    % Text Node
    \draw (283,193.97) node [anchor=north west][inner sep=0.75pt]   [align=left] {$\displaystyle \vdots $};
    % Text Node
    \draw (322.5,193.97) node [anchor=north west][inner sep=0.75pt]   [align=left] {$\displaystyle \vdots $};
    % Text Node
    \draw (363,193.97) node [anchor=north west][inner sep=0.75pt]   [align=left] {$\displaystyle \vdots $};
    % Text Node
    \draw (420.1,194) node [anchor=north west][inner sep=0.75pt]   [align=left] {$\displaystyle \vdots $};
    % Text Node
    \draw (382.4,194.52) node [anchor=north west][inner sep=0.75pt]   [align=left] {$\displaystyle \ddots $};
    % Text Node
    \draw (292,45) node [anchor=north west][inner sep=0.75pt]   [align=left] {${Heat\ Zone}$};
    % Text Node
    \draw (220.87,45.67) node [anchor=north west][inner sep=3pt]   [align=left] {\large {$\displaystyle \mathtt{(a)}$}};
    % Text Node
    \draw (490.67,121.47) node [anchor=north west][inner sep=3pt]   [align=left] {\large {$\displaystyle \mathtt{(b)}$}};
    % Text Node
    \draw (492.43,150.41) node [anchor=north west][inner sep=0.75pt]   [align=left] {{\tiny {\fontfamily{ptm}\selectfont p}}};
    % Text Node
    \draw (479.75,168.95) node [anchor=north west][inner sep=0.75pt]   [align=left] {{\fontfamily{ptm}\selectfont {\tiny q}}};
    % Text Node
    \draw (492.83,188.12) node [anchor=north west][inner sep=0.75pt]   [align=left] {{\tiny {\fontfamily{ptm}\selectfont r}}};
    % Text Node
    \draw (525.34,167.95) node [anchor=north west][inner sep=0.75pt]   [align=left] {{\tiny {\fontfamily{ptm}\selectfont q'}}};
    % Text Node
    \draw (504.00,168.45) node [anchor=north west][inner sep=0.75pt]   [align=left] {{\tiny {\fontfamily{ptm}\selectfont A}}};
    
    \draw [<->] [line width=1.0] (283.2, 295.4) -- (365.2, 295.4) ;
    \draw [->] [line width=1.0] (310, 73) -- (337, 73);
    % \draw   (283.2,286.4) .. controls (283.2,291.07) and (285.53,293.4) .. (290.2,293.4) -- (314.2,293.4) .. controls (320.87,293.4) and (324.2,295.73) .. (324.2,300.4) .. controls (324.2,295.73) and (327.53,293.4) .. (334.2,293.4)(331.2,293.4) -- (358.2,293.4) .. controls (362.87,293.4) and (365.2,291.07) .. (365.2,286.4) ;    \draw   [->] (312.63,73.33) -- (336.63,73.33) ;
    % Text Node
    \draw (319.33,63.67) node [anchor=north west][inner sep=0.75pt]   [align=left] {$\displaystyle v$};
    % Text Node
    \draw (306.20,298.47) node [anchor=north west][inner sep=0.75pt]   [align=left] {$\displaystyle Width$};

    
    \end{tikzpicture}
    \caption{Surface programmable thermodynamic computer (TC) and rule110 automaton gadget. $\mathtt{(a)}$ Example of a surface programmable TC with a moving heat bath slowly doing ``calculation''. White circles are spins and black lines represent interaction between spins, while red rectangles represent cellular automaton gadget. $\mathtt{(b)}$ Example of a rule110 gadget. Here spins $p$, $q$, and $r$ represent the left, middle, and right states in the cellular automaton, respectively, while $q’$ represents the result computed by the cellular automaton, and $A$ is an auxiliary spin. We define a spin pointing up as representing logical 1 and pointing down as representing logical 0. }
    \label{Surface-programmable-thermodynamic computer}
\end{figure}

The physical implementation of the TC is based on spins, with computational information encoded in the orientation of the spins, represented by white circles, while the red box represents a logic circuit, composed of several spins. The horizontal axis represents the number of computational layers, with spins belonging to the same computational layer sharing the same horizontal coordinate, and vertical axis represents the number of spins in each layer. 

The input layer consists of the leftmost spin, while the output layer consists of the rightmost spin. The entire computing process can be described by gradually moving the external heat bath, which creates a temperature gradient and, consequently, a heat zone, from the input layer to the output layer. By utilizing thermal fluctuations from the heat bath, the spins in different layers sequentially settle into low-energy states within the state space, effectively completing the iterative computation of the cellular automaton.

Users can design the computing algorithm and obtain the desired output by setting the input configuration and then making slight movements to the external heat bath, requiring only approximately linear time (up to a $\log^2$ factor) with respect to the number of layers.

\section{Model and Method}




\subsection{Elementary cellular automaton}\label{elementary-cellular-automaton}
\ym{I haven't made change in this section}

An elementary cellular automaton is a 1-dimensional cellular
automaton, where there are two possible states (labeled by 0 and 1). The rule to
determine the state of the cell in next generation depends only on the
current state of the cell and its two immediate neighbors.

There are $8 = 2^3$ possible configurations for a cell and its two
immediate neighbors. Different elementary cellular automaton are only
different from their translation rules. There are only $2^8 = 256$
different rules, so do the automatons.

If we put each possible current configurations in order: 111, 110,
\ldots, 001, 000, and put the resulting state under them. We then get an
integer in its binary representations. Then this integer is taken to be
the rule number of the automaton. For example, rule 110.

Given that $110_d = 01101110_2$, so rule 110 is defined by the
translation rule:

\begin{tabular}{|c|c|c|c|c|c|c|c|c|}
\hline
Current Pattern & 111 & 110 & 101 & 100 & 011 & 010 & 001 & 000 \\
\hline
New Center Cell & 0 & 1 & 1 & 0 & 1 & 1 & 1 & 0 \\
\hline
\end{tabular}

Rule 110 has been shown to be Turing Complete~\cite{Cook2009}, and thus capable of universal
computation.

\subsection{Mapping CA to Ising model}

A spin-glass hamiltonian is described as follows.

\begin{equation}
H = \sum_{u,v \in E} J_{u,v}s_us_v + \sum_{i\in V}h_i s_i
\end{equation}

where $h_i$ is onsite energy, $J_{u,v}$ is interaction energy and
$s_i \in \{-1, 1\}$ is local spin. $s_i=1$ means spin-up and $s_i=-1$ means spin-down.

Previous studies already shown that it is possible to encode a universal set of elementary logic gates into Ising models and concatenate them to formulate any Boolean truth table in Ising models~\cite{onizawa2020design, camsari2017stochastic, Aadit2022}. However, we consider using this method to encode Boolean truth tables to be inefficient, as the interaction complexity on each spin is relatively low. In Appendix \ref{sec:gadget-design}, we used a linear programming method to directly encoding Boolean truth tables into the ground state of the Ising model. This method is applied to compute the model corresponding to the rule 110 automaton, which we refer to as the gadget, shown in [Fig. \ref{Surface-programmable-thermodynamic computer}(b)].

The interaction energy and onsite energy of the rule 110 gadget are given below. Spins $1, 2, 3$, and $4$ correspond to the $p, q, r$, and $q'$ spins in [Fig. \ref{Surface-programmable-thermodynamic computer}(b)], while spin $5$ is the auxiliary spin.

\begin{equation}
J = \begin{pmatrix}
~\cdot~ & ~1~ & ~1~ & ~2~ & ~3~\\
\cdot & \cdot & 2 & 2 & 5\\
\cdot & \cdot & \cdot & 2 & 5\\
\cdot & \cdot & \cdot & \cdot & 6\\
\cdot & \cdot & \cdot & \cdot & \cdot
\end{pmatrix}, h = \begin{pmatrix}
~1~\\
2\\
2\\
2\\
5
\end{pmatrix}
\end{equation}

Since a copy gate can be simply described as two spins with an interaction $J_{u, v}=1$, and by applying the method of connecting logic gates~\cite{Aadit2022}, we can concatenate the rule 110 gadget with some copy gates, ultimately forming the TC shown in [Fig. \ref{Surface-programmable-thermodynamic computer}(a)]. The gadget we constructed based on the Rule 110 cellular automaton
naturally possesses Turing completeness. Therefore, this 2-dimensional TC is capable of universal computing.

\subsection{Cooling computation with a moving heat bath}\label{cooling-computation-with-moving-heat-bath}

The computational process of the cellular automaton is divided into three steps: determining the state of each input cell, iteratively applying the transition rules to generate the next generation of cells, and identifying a certain generation of cells as the output. In fact, the width of the time axis in the Ising TC corresponds to the number of iterations. Hence, the smallest layer of spins on the time axis should be the input layer, and the largest layer of spins should be the output layer. 

Given that each ground state of the TC corresponds to a valid computational process of the cellular automaton for a given input, the computational process of the TC is essentially reduced to finding the ground state of the TC with the spin configuration in the input layers fixed.

We focus here on the thermodynamics-driven approach to finding ground states, specifically the heating-annealing method. This involves gradually increasing and then decreasing the temperature of the heat bath with which the TC interacts. The TC may employ heat fluctuations to explore a state space in the first stage and stabilize on ground state during the second stage as long as the annealing stage is long enough. However, it is convinced that this stage can last for exponential time for some hard spin-glass problems.

Due to the structure of the TC, we can allow only a portion of the system, rather than the entire system, to interact with the heat bath, while ignoring the interaction energy between the boundary spins and the exterior. After completing the heating-annealing procedure on one part of the system, we move the heat bath to another part and repeat the same process. This approach effectively finds a local ground state for each part of the system, which is reasonable when the global ground state can be divided into multiple local components.

More specifically, the computation, progressing from the input layer to the output layer, involves the following steps: (1) Initialize the input layer configuration by adjusting the onsite energy for each spin. (2) Determine the number of layers $W$ that will interact with the external heat bath in each heating-annealing process. Group the layers in sets of $W$ from input to output, labeling them as groups $1, 2, \cdots, \frac{T}{W}$. The number of heating-annealing repetitions will then be $\frac{T}{W}$, where T is the total number of layers. (3) Perform the heating-annealing process on each group sequentially.

However, solving the ground state from the non-deterministic direction is at least as hard as solving the circuit satisfiability problem, which is NP-complete~\cite{Moore2011}.

% \subsection{Cooling computation with direction}\label{sec:direction}

% Definitions: * \emph{in-surface/out-suface}: The surface of a surface
% programmable material that associated with the input/output of the logic
% circuit.

% The process of heating and annealing the entire material does not have computational capability, as the system lacks logical directionality.

% Logical directionality needs to be introduced through an orderly annealing method. 
% Similar to the localized heating-annealing approach that catalyzes the growth of ordered crystals, we introduced a temperature gradient annealing scheme.

% More specifically, the computation, with computing direction from in-surface to out-surface, contains the following steps: (1). Initialize the in-surface configuration. By removing
% some atoms on the in-surface. (2). Connect the in-surface/out-surface to
% external heat sources at temperature $T_1 < T_2$, respectively. We
% also require that the energy gap between the ground state and the first
% excited state of the Hamiltonian to be $\Delta E< T_1$. (3). Lower the
% temperature of the heat sources ``slowly'' to cool the system to the
% ground state of the Hamiltonian. The temperature of the heat sources at
% time $t$ is $T_{1/2}(t) = T_{1/2}(0)\lambda^{-c t}$, where
% $T_{1/2}(0)$ is the initial temperature of the heat sources, and
% $\alpha$ is a constant.

% The computing direction could be reversed by swaping the temperature of the heat sources connected in-surface and out-surface.
% However, solving the ground state from the non-deterministic direction is at least as hard as solving the circuit satisfiability problem, which is NP-complete~\cite{Moore2011}.

\subsection{Computing by cooling}\label{local-cooling}

We test the hypothesis: Cooling is easy if the process is from the
deterministic direction, hard if the process is from the
non-deterministic direction.

\begin{hyp}
Let $\mathcal{P}$ be a circuit SAT problem, and $\mathcal{E}$ be an energy modeling encoding of $\mathcal{P}$ (locally). There exists a subset of $\mathcal{E}$ that exhibits the overlap gap property~\cite{Gamarnik2021}.
\end{hyp}
The overlap gap property is a property for describing the algorithmic intractability in random structures, which is based on the topological disconnectivity property of the set of pair-wise distances of near optimal solutions.
If the energy landscape of a optimization problem exhibits the overlap gap property, then it can not be solved by any local algorithms in polynomial time, including the simulated annealing method used in this work.
The intuition comes from the energy model encoding of the identity circuit.
Given a circuit with $n$ bits, the ground state degeneracy of the corresponding energy model is at least $2^n$.
The energy model can be vertically stacked $n$ copy gadgets.
Let the circuit depth be $d \sim 2^{\gamma \sqrt{n}}$ for some constant $\gamma > 0$.
The Hamming distance between two degenerate ground states for different input signals is at least $d$, which is exponential to $n$. Let us define the size of the problem as $d\times n$, then this distance is clearly linear to the problem size. We left the rigorous proof of this hypothesis to mathematicians.

In our gadget, cooling from deterministic direction is from input to
output, non-deterministic direction is from output to input. The latter
one must be non-deterministic because this gadget is Turing-Complete.

The goal is to drive the the \textbf{state} to the one with lowest energy as fast as possible, while some bits set to be 1 or 0 in the \textbf{state}.

\subsubsection{Estimation of the computing time along computation direction}\label{estimation-of-the-computing-time}

\ym{This part should be rewrote in future..My words are so poor.}

Let us initially treat the heat bath as ideal, assuming that the temperature within the interacting region of the system is uniform and that the temperature outside this region is zero. We refer to the interacting region as the ``Heat Zone''.

Assume there is an annealing procedure that may prevent the layers in the Heat Zone from stabilizing in the local ground state, or equivalently, that an error occurs in the computation process with a probability $\epsilon$. Repeating this procedure $\frac{T}{W}$ times would result in a probability of $(1-\epsilon)^{\frac{T}{W}}$ to successfully find the proper global ground state. To maintain a constant success probability as $T \rightarrow +\infty$, the error probability $\epsilon$ should scale as $\frac{W}{T}$.

Now we turn to determining the annealing time $t_{th}$ required to achieve an error probability $\epsilon$ in each Heat Zone. Let $\lambda_1$ and $\lambda_2$ be the largest and second largest eigenvalues of the Markov transition matrix, respectively. The relationship between $\frac{\lambda_2}{\lambda_1}$ and $W$ can be expressed as $1 - \frac{\lambda_2}{\lambda_1} \sim e^{-W}$. Consequently, a relationship exists between $t_{th}$ and $\epsilon$, which is given by:

\begin{equation}
    \left( \frac{\lambda_2}{\lambda_1} \right) ^ {t_{th}} = \epsilon \rightarrow t_{th} \sim \ln(\frac{T}{W})e^{W}
\end{equation}

And the total annealing time scale as $\frac{T}{W}\ln(\frac{T}{W})e^W$.

In practice, temperature changes cannot occur instantaneously, and the temperature distribution within the heating zone may not be uniform. Thus, it is necessary to account for the effective width of the Heat Zone when the the heat bath are not ideal.

Suppose the middle of the heat bath is at $middle(t)$, and the temperature at the region with a Time coordinate $x$ can be represent as $Temp(x) \propto \Lambda^{-|x - middle(t)|}$. Then

\begin{equation}
    e^{-\Delta E \Lambda^{cW}} \sim \epsilon \sim \frac{W}{T}
\end{equation}

Where $c$ is the empirical constant for effective width and can be determined during the experiment. The total annealing time should be

\begin{equation}
    t_{total} = \frac{T}{W}(\ln(\frac{T}{W}))^{1+ \frac{1}{c\ln(\Lambda)}}
\end{equation}

\subsubsection{Estimation of the computing time along input memory direction}\label{Estimation-of-the-computing-time-along-input-memory-direction}

We fixed the number of layers and denote $n$ as the width of each layer in this material.

However, we fail to derive an explicitly expression for the time complexity in this direction. 
Numerical result shows that computing complexity is linear to $n$. Thus a total time
for the computation should scale as $nm\log^2(m)e^{(1-\lambda)^{-1}}$ 

\subsubsection{Testing model through equilibrilium SA}\label{temperature-gradient-and-energy-gradient}

We used simulated annealing to mimic the heating-annealing process.
However, classical simulated annealing can only handle the situation
where the temperature is space-invariant. Thus, a scheme change is needed.

One can simply
make the following modification to the transition probability in the
markov process and return to the situation, called energy-gradient scheme, where the temperature is space-invariant.

\begin{equation}
e^{-\frac{\Delta E_k}{T\lambda^{ct+k}}} = e^{-\frac{\Delta E_k(\frac{1}{\lambda})^k}{T\lambda^{ct}}} = e^{-\frac{\Delta E_k \Lambda^k}{T(t)}}
\end{equation}

Where we denote $\Lambda = \frac{1}{\lambda}$ and the time complexity in $\Lambda$ form is apparently
$T_{total} \sim \frac{m}{W}\log^2(\frac{m}{W})e^{(\Lambda-1)^{-1}}$



\subsection{Algorithm beyond temperature
gradient}\label{algorithm-beyond-temperature-gradient}

Applying a huge energy gradient or temperature gradient to this material
seems not a good idea when the system become larger.

Inspired by the way clothes are ironed, we conceived an idea called wave-packet scheme, that slowly
scanning a wave-shaped temperature curve from one end to the other. This
method is widely used in studying the heating-recrystallization
properties of real materials~\cite{Zhang2014}.

\begin{figure}[h]
\centering
\includegraphics[width=\columnwidth]{../notes/images/zhang2014.png}
\caption{Method to scan the wave packet}
\end{figure}

There are various specific choices for the wave packet, such as
power-law, exponential, and others. In this context, we primarily focus
on a exponential wave packet, which can be directly linked to the
concept of energy gradient.

A expoential wave packet with central position $middle(t)$, amplitude
$A$ and base $\Lambda$, can be express as following.

\[
T(k,t) = A\Lambda^{|k-middle(t)|}
\]

So slowly moving the wave packet means slowly changing $middle(t)$

Let's review the concept \textbf{active zone} defined earlier. When the
temperature of the system is defined as
$T_{tg}(k, t) = T\lambda^{ct-k}$, atoms in the active zone are those
whose layer $k$ belongs to $k<ct - \frac{W}{2}$, where $W$ are
given by
$e^{-\frac{\Delta E_{max}}{\lambda^{\frac{W}{2}}}}\leq \epsilon$. For
those $k<ct - \frac{W}{2}$, the probability that SA would flip an
atom's state is no more than
$e^{-\frac{\Delta E_{max}}{T\lambda^{ct}\lambda^{-ct + \frac{W}{2}}}}=e^{-\frac{\Delta E_{max}}{T\lambda^{\frac{W}{2}}}}\leq \epsilon^{\frac{1}{T}}$,
which means these previous layer would stay ``frozen''. For those layer
much deeper than $k = ct$, SA would just randomly flip them. These
layer won't affect layers in active zone.

Now turn our sight into the wave packet model, we can construct a
similar concept \textbf{heat zone}. Layers in the heat zone are those
$A\lambda^{|k-middle(t)|}\ge \epsilon\rightarrow middle(t) + \frac{\log(\epsilon)}{\log(\Lambda)}< k < middle(t) - \frac{\log(\epsilon)}{\log(\Lambda)}$.

The layers that shallower than
$middle(t) + \frac{\log(\epsilon)}{\log(\Lambda)}$ correspond to
layers shallower than $ct-\frac{W}{2}$ in temperature gradient model,
since they are both stay ``frozen''; layers between
$middle(t) +\frac{\log(\epsilon)}{\log(\Lambda)}$ and $middle(t)$
are effectively cooled down, which correspond to the \textbf{active
zone}; layers deeper than $middle(t)$ are either randomly fliped or
stay ``frozen'', but we don't care their state.

Therefore, the wave packet model are just the same as the temperature
gradient model, and is more likely to be realized physically.

\subsubsection{Some details about non-equilibrium simulated
annealing}\label{some-details-about-non-equilibrium-simulated-annealing}

The classical simulated annealing can only deal with the situation that
the temperature is space-invariant. Although there are many ways to
simulated a real-world system with temperature gradient, such as
Molecular Dynamics \cite{Bai2015, Deng2006}, but
these methods mainly depend on dynamically adjusting the atomic
velocities to achieve a temperature gradient, which can't be directly
used in our model.

There are also Monte Carlo-based methods to simulate a moving
temperature wave packet that facilitates material growth~\cite{Godfrey1995, Tan2017}.
However, these methods fail to establish a direct connection between the simulated temperature and experimental
temperature~\cite{Zollner2014}.
Therefore, based on HeatBath acceptance rule, we made a slight modification to the classical SA model.

When we flip one atom's state, there would be some energy difference
$\Delta E_{i}$ in field $i$ with temperature $T_i$. The
probability to accept this flip would be

\begin{equation}
P = \frac{1}{1 + e^{\sum_i \frac{\Delta E_i}{T_i}}}
\end{equation}

It is easily to observe that when the system returns to
heat-equilibrium, the probability reverts to the classical HeatBath
acceptance probability.
\section{Numerical simulation}\label{sec:numerical-result}

\subsection{Energy model}\label{a-2d-surface-programmable-material}

Given that simulating a cellular automaton with a deep computational layer on the Ising model requires a large number of spins, making large-scale numerical simulations impractical, we opted to introduce a similar but simpler energy model to observe the behavior on a larger scale.

Let us define a model Hamiltonian on a 2D lattice of size $S \times T$ as
\begin{equation}\label{eq:toy-hamiltonian}
H = \sum_{t=1}^{T-1}\sum_{i=1}^{S} (n_{i+1, t} \land \neg(n_{i, t} \land \neg n_{i-1, t})) \veebar n_{i,t+1},
\end{equation}
where $n_{i, t} \in \{0, 1\}$ is the state of the cell at row $i$ and column $t$, the boundary is periodic in the space direction, i.e. $n_{0, t} = n_{S, t}$.
Its ground state with energy $0$ encodes a $110$ cellular automaton, where the direction of computation is from the left to right.
\subsection{Numerical result}


We conducted numerical experiments both for energy-gradient and wave-packet scheme, under the condition
$\Lambda = 1.3, n = 15$ , and fitted each sweep time against
$m \log^2(m)$. Where $m$ represents the number of the single layers,
$n$ represents the width per single layer.

\begin{figure}[h]
    \centering
    \includegraphics[width=\columnwidth,keepaspectratio]{../notes/images/numercial result.pdf}
    \caption{Time v.s. Depth of the material}
\end{figure}

Another numerical experiment are conducted to determine the time complexity of the computation along the input memory direction. 

\begin{figure}[H]
    \centering
    \includegraphics[width=\columnwidth,keepaspectratio]{../notes/images/toymodel_time_vs_width.png}
    \caption{Time v.s. width of the material}
\end{figure}

This indeed informs that the time scale along the memory direction is linear.

\section{Physical implementation on Ising machine}\label{spin-glass}

To examine whether this model could achieve a better performance when
using adiabatic methods, we need to generalize this energy function to
some good hamiltonian. There has already been work on mapping
combinatorial optimization problems to a spin-glass model (known as the
Ising Machine) and leveraging the properties of digital devices that
simulate this Ising Machine to find an appropriate ground state, which
encodes computational information~\cite{Aadit2022,Bybee2023}.



\section{Discussion and Outlook}
Connect real-world thermalization time and the simulated annealing time to explore the connection between time and energy cost per computation.
The emergence of wisdom?

\begin{acknowledgments}
    We thank Lei Wang, Madelyn Cain and XXX for helpful discussions on the simulation methods.
\end{acknowledgments}

%\bibliographystyle{apsrev4-1}
\bibliography{refs.bib}

\appendix

\section{Ising machine gadget design}\label{sec:gadget-design}
To map our toy model in \Cref{eq:toy-hamiltonian} to
spin-glass model, we used Linear Programming algorithm.
Linear problems are problems that can be express in standard form as

\begin{equation}
    \begin{split}
        &\min_{x \in \mathbb{R}^n} \sum_{i=1}^n c_ix_i\\
        &\text{s.t. } l_j \leq \sum_{i=1}^n a_{i,j}x_i \leq u_j, \; j=1,\ldots,m\\
        &p_i \leq x_i \leq q_i, \; i=1,\ldots, n
    \end{split}
\end{equation}

Here with a spin-glass consists $N$ atoms, we choose $n$ to be $\frac{N(N-1)}{2} + N$, which means each variable $x_i$ represents a correspond interaction energy $J_{u, v}$ or a correspond onsite energy $h_u$, we assigned aliases to these $x_i$ as $x_{u,v}$ or $x_{u}$.

Let the configuration space of the system be $S = \{\mathbf s_i\mid i=1,\ldots, 2^N\}$, each associated with an energy $H(\mathbf s_i)$. We denote the target states with minimum energy as $S_{\text{min}} \subseteq S$. We can express the linear programming problem as

\begin{equation}
    \begin{split}
        &\min_{J \in \mathbb{R}^{N(N{-}1)/2}, h\in \mathbb{R}^N} 0\\
        &H(\mathbf s_i) < H(\mathbf s_j), \forall \mathbf s_i \in S_{\text{min}}, \mathbf s_j \in S \setminus S_{\text{min}}\\
        &H(\mathbf s_i) = H(\mathbf s_j), \forall \mathbf s_i, \mathbf s_j \in S_{\text{min}}
    \end{split}
\end{equation}

Note that $H$ is a linear function of $J$ and $h$, so the
constraints are linear. The less constraints and equality constraints
can be easily transformed into inequality constraints by adding ancilla
variables.

There is no solution with 4-atoms spin-glass that can satisfy the
constraints. The minimum number of atoms required to satisfy the
constraints is 5. We set the target states to

\begin{table}[H]
    \centering
\begin{tabular}{|c|c|c|c|c|}
\hline
input 1 & input 2 & input 3 & output & ancilla \\
\hline
$\downarrow$ & $\downarrow$ & $\downarrow$ & $\downarrow$ & $?$ \\
$\downarrow$ & $\downarrow$ & $\uparrow$ & $\uparrow$ & $?$ \\
$\downarrow$ & $\uparrow$ & $\downarrow$ & $\uparrow$ & $?$ \\
$\downarrow$ & $\uparrow$ & $\uparrow$ & $\uparrow$ & $?$ \\
$\uparrow$ & $\downarrow$ & $\downarrow$ & $\downarrow$ & $?$ \\
$\uparrow$ & $\downarrow$ & $\uparrow$ & $\uparrow$ & $?$ \\
$\uparrow$ & $\uparrow$ & $\downarrow$ & $\uparrow$ & $?$ \\
$\uparrow$ & $\uparrow$ & $\uparrow$ & $\downarrow$ & $?$ \\
\hline
\end{tabular}
\caption{The eight degenerate ground states of the spin-glass model to implement the $110$ rule.}
\end{table}

where the ancilla bit in each row can be either $\uparrow$ or
$\downarrow$ (256 total possibilities). One of the solutions
satisfying the constraints is

\begin{equation}
J = \begin{pmatrix}
~\cdot~ & ~1~ & ~1~ & ~2~ & ~3~\\
\cdot & \cdot & 2 & 2 & 5\\
\cdot & \cdot & \cdot & 2 & 5\\
\cdot & \cdot & \cdot & \cdot & 6\\
\cdot & \cdot & \cdot & \cdot & \cdot
\end{pmatrix}, h = \begin{pmatrix}
~1~\\
2\\
2\\
2\\
5
\end{pmatrix}
\end{equation}

\section{The implmentation in Rydberg atoms array}\label{physical-model-rydberg-atoms-array}
Let us consider a Rydberg atoms array with a two dimensional embedding. The coordinate of the atom $v$ is denoted as $\mathbf{r}_v$.
The Rydberg Hamiltonian~\cite{Nguyen2023} is defined as

\begin{equation}
    H_{\text{Ryd}} = \sum_v \dfrac{\Omega_v}{2} \sigma^x_v -\sum_v \Delta_v n_v + \sum_{v < w}  V_{\text{Ryd}}(|\overrightarrow{\mathbf{r}_v} -\overrightarrow{\mathbf{r}_w}|)n_v n_w,
\end{equation}
where $\Omega_v$ is the Rabi frequency, $\Delta_v$ is the detuning,
$n_v = \dfrac{1}{2}(1 - \sigma^z_v)$ is the number operator, and
$V_{\text{Ryd}}(|\overrightarrow{\mathbf{r}_v} - \overrightarrow{\mathbf{r}_w}|) = C_6/|\overrightarrow{\mathbf{r}_v} - \overrightarrow{\mathbf{r}_w}|^6$
is the Rydberg interaction potential.

In the following, we will show the classical part of the Rydberg Hamiltonian encodes an independent set problem on a unit disk graph.
In graph theory, an independent set is a set of vertices in a graph, no two of which are adjacent.
Here, we restrict the graph to be a two dimensional unit disk graph $G=(V, E)$, a geometric graph with two dimensional embedding, such that two vertices $u, v \in V$ are connected if and only if they are within the unit radius, i.e. $|\mathbf r_u - \mathbf r_v| < r_B$. This reflects the blockade phenomenon in Rydberg atoms arrays, where two atoms within blockade radius $r_B$ can not excite to the Rydberg state simultaneously due to the strong Rydberg interaction potential. Let us idealize the energy model by treating the blockade interaction between two atoms as infinity if two atoms are within blockade radius, zero otherwise. Then the energy model becomes
\begin{equation}
H_{\text{MWIS}} = -\sum_{v \in V}\Delta_v n_v + \sum_{(u, v) \in E} \infty n_u n_v.
\end{equation}
The ground state of which encodes the maximum weight independent set (MWIS) problem, and $\Delta_v$ is the weight associated with vertex $v \in V$. Whenever we have a atom in the Rydberg state in the ground state of $H_{\text{WMIS}}$, we add the vertex associated with that atom to the maximum independent set.

\subsection{Gadget finding}
Given a target logic gate, the goal is to find a graph $G = (V, E)$ and weights associated with vertices $\{\Delta_v \mid v\in V\}$ such that the WMIS problem of this graph encodes this gate.

Let $\mathcal{M}$ be the set of all maximal independent sets of $G$, where a maximal independent set is an independent set that is not a subset of any other independent set.
Each $\mathbf n \in \mathcal{M}$ is associated with an energy $H_{\text{WMIS}}(\mathbf n) = -\sum_v \Delta_v n_v$. We denote the target states with minimum energy as $\mathcal{M}_{\text{min}} \subseteq \mathcal{M}$. We can express the linear programming problem as

\begin{equation}
    \begin{split}
        &\min_{\boldsymbol{\Delta} \in \mathbb{R}^{|V|}} 0\\
        &\sum_i n_i' \Delta_i < \sum_i n_i \Delta_i, \forall \mathbf n \in \mathcal{M}_{\text{min}}, \mathbf n' \in \mathcal{M} \setminus \mathcal{M}_{\text{min}}\\
        &\sum_i n_i \Delta_i = \sum_i n_i' \Delta_i, \forall \mathbf n, \mathbf n' \in \mathcal{M}_{\text{min}}
    \end{split}
\end{equation}

Note that the constraints are all linear. The less constraints and equality constraints can be easily transformed into inequality constraints by adding ancilla variables.
The possible choice of $S_{\text{min}}$ grows exponentially as the number of ancilla vertices increases, which limits the capability of this method.
\jgl{We focus on the general graphs first.}

In the following, we restrict the graph to be two dimensional grid graphs, the unit distance can be arbitrary. A graph configuration can be denoted as a boolean mask $M\in \{0, 1\}^{m\times n}$.
\jgl{Then the unit disk graphs.}

\subsubsection{Energy based universal computation with Rydberg atoms
array}\label{energy-based-universal-computation-with-rydberg-atoms-array}

\textbf{Statement 3}: The classical Rydberg Hamiltonian is universal for classical computation.

The NOR gate can be implemented using the Rydberg Hamiltonian (subfigure
c below). The NOR gate is a universal gate for classical computation.
\includegraphics[width=\columnwidth]{../notes/images/gadgets.png}

The conjunction of gates can be implemented by ``gluing'' the Rydberg
atoms together (subfigure d below). The weights are added together.

For more logic gates, please check the GitHub repository
\href{https://github.com/QuEraComputing/UnitDiskMapping.jl/blob/main/test/logicgates.jl}{UnitDiskMapping.jl}.

\subsubsection{Cooling the Rydberg Hamiltonian}\label{cooling-the-rydberg-hamiltonian}

\textbf{Statement 4}: The Rydberg Hamiltonian, if cooled successfully
with some vertices fixed to certain configuration, can be used to solve
the circuit satisfiability problem, which is NP-complete~\cite{Moore2011}.

$P \neq NP$: Cooling is generally hard, especially when from the
non-deterministic direction.

\subsubsection{The Rule 110 Gadget}\label{the-rule-110-gadget}

We can encode the Rule 110 cellular automaton into a Weighted Maximum
Independent Set Problem, with blue vertices assigned a weight of 1 and
red vertices assigned a weight of 2, as follows.

\includegraphics[width=0.6\columnwidth]{../notes/images/image.png}

This graph can be embedded into a grid graph, where two vertices are
connected if and only if their Euclidean distance is no more than $2$.

\includegraphics[width=0.7\columnwidth]{../notes/images/image-1.png}

The correspondence between the Maximum Weighted Independent Set (MWIS)
Solution and Rule 110 is as follows:

The states of vertex \textbf{1}, vertex \textbf{3}, and vertex
\textbf{8} represent the states of the \textbf{middle}, \textbf{left},
and \textbf{right} cells of the automaton's \textbf{input},
respectively. If the input value of a cell is 1, then the corresponding
vertex must be in the MWIS solution; otherwise, it is not. Vertex
\textbf{12} corresponds to the automaton's \textbf{output}. If the
automaton output is 1, then vertex 12 is in the MWIS solution;
otherwise, it is not.

In the automaton diagram, the above gadget is equivalent to:

\begin{figure}
\centering
\includegraphics[width=\columnwidth]{../notes/images/rule110.png}
\caption{Alt text}
\end{figure}

There are exactly \textbf{8} different MWIS solutions in this graph (the
weighted size of each MWIS solution is 7), each corresponding to one of
the \textbf{8} possible outputs of the automaton. We list them as
follows.
\includegraphics[width=\columnwidth]{../notes/images/gadget110.png}

Utilizing copy gadget and cross gadget~\cite{Nguyen2023}, we construct a \textbf{Surface Programmable Material} with open boundary conditions as follows.

\begin{figure}
\centering
\includesvg[width=\columnwidth]{../notes/images/rule110transvarient.svg}
\caption{Alt text}
\end{figure}

The above gadget depicts a two-layer cellular automaton. The vertices in
blue, red, green and black have weights of 1, 2, 3 and 4, respectively.
In the automaton diagram, the above gadget is equivalent to:

\begin{figure}
\centering
\includegraphics[width=\columnwidth]{../notes/images/rule110_2-2_automaton.png}
\caption{Alt text}
\end{figure}

\subsection{Speed and work}\label{speed-and-work}

The trade-off between the energy consumption and the speed of
computation~\cite{Feynman2018}. To avoid confusion, we emphasize the
``energy consumption'' is defined as the work done in a computational
process, which is the same as the amount of heat dissipated to the
environment. This quantity has a lower bound given by the Landauer
principle, which states that the work done in a computation is at least
$kT\ln 2$ per bit erased\cite{Reeb2014}.

Information erasure in the surface programmable material is proportional
to the volume of the material, which is $O(tS)$, where $t$ is the
time of computation, and $S$ is memory (proportional to the surface
area) of the material.

From the chemical reaction perspective, the speed of computation is
determined by the parameter $\lambda$.

\subsubsection{Quantum adiabatic annealing energy gap}\label{quantum-adiabatic-annealing-energy-gap}

One possible way is to use quantum adiabatic annealing: start from a
simple hamiltonian $H(0)$ and its simple ground state
$|\psi(0)\rangle$, then gradually change the parameters until reaching
the desire hamiltonian $H(t)$.

More specifically, set $\Delta(t=0) <0$ and $\Omega(t=0) =0$
initially, then first turning on $\Omega(t)$ to a non-zero value,
sweeping $\Delta(t)$ to final value, and finally turning off
$\Omega(t)$.

\begin{equation}
H_{QAA}(t) = \sum_{v\in V} (-\Delta(t)w_v \hat n_v + \Omega(t)\sigma_{v}^x) + \sum_{(u,w) \in E} U\hat n_u \hat n_w
\end{equation}

If the time evolution is sufficiently slow, then by the adiabatic
theorem, the system follows the instantaneous ground state, ending up in
the solution to the MWIS problem~\cite{Pichler2018}.
Then we only need to evalute
the minimum energy gap $\Delta_{QAA}$ between the ground and
first-excited states of instantaneous hamiltonian.

We set $\Omega = 1 \times 2\pi$ and sweep the $\Delta$ from $3
\times 2\pi$ to $40 \times 2\pi$ with 1*1 gadget. For deterministic
direction, we simply set the weight of the input vertices to $50$; as
for non-deterministic direction, we set the weight of the output vertice
to $50$.

Result listed as follows. \textbf{However, we didn't see cooling from
deterministic direction would give a smaller energy gap than the other
direction. We think that's because the size of this gadget is too
small.}

\begin{figure}
\centering
\includegraphics[width=\columnwidth]{../notes/images/energy_gap_1_gadget.png}
\caption{Alt text}
\end{figure}

\section{Classical Dynamics}\label{classical-dynamics-1}
We firstly tried classical adiabatic annealing with the classical-spin
mapping method~\cite{Wang2013}
\begin{equation}
\frac{\partial \vec M_i}{\partial t} = \vec M_{i} \times \vec H_{i}(t)
\end{equation}

where

\begin{equation}
\vec H_i(t) = -\frac{\partial H(t)}{\partial \vec M_i} 
\end{equation}

Here the hamiltonian of the system is

\begin{equation}
H(t) = \frac{t}{T}(\sum_{u,v}J_{u,v} M_{u,z}M_{v,z} + \sum_{u} h_u M_{u, z}) + (1-\frac{t}{T})(I\sum_{u}M_{u,x})
\end{equation}

The former term is the generally increasing target hamiltonian, the
latter term is a generally decreasing known Ising-transverse field. We
can explicitly write out the effective magnetic field.

\[
\vec H_{i}(t) = -\frac{t}{T}(\sum_{v}J_{i, v}M_{v,z} + h_i)\hat e_z - (1-\frac{t}{T})I\hat e_x
\]

Integrate the ordinary differential equation then we get the classical
dynamics of this spin-glass model. However, result shows that it is
extremely hard to find the solution when the number of layers exceed
$4$ (input layer is pinned).

The reason maybe that the mapped spin-glass model lies in the hard
region in ~\cite{Wang2013}. Because every interaction energy is positive, which
gives rise to strong frustration. Results in this reference also shows that even in
random spin-glass model, there exist hard instance that can't be solve.



\end{document}
